\documentclass[12pt,a4paper,magyar]{article}
\usepackage[utf8]{inputenc}
\usepackage[T1]{fontenc}
\usepackage[margin=1in]{geometry}
\usepackage{xcolor}
\usepackage{listings}
\usepackage{hyperref}
\usepackage{fancyhdr}
\usepackage{graphicx}
\usepackage{array}
\usepackage{booktabs}
\usepackage{babel}

\lstset{
    language=C,
    basicstyle=\ttfamily\small,
    keywordstyle=\color{blue},
    commentstyle=\color{gray},
    stringstyle=\color{red},
    breaklines=true,
    showstringspaces=false,
    tabsize=4,
    frame=single,
    backgroundcolor=\color{gray!10},
    numbers=left,
    numberstyle=\tiny
}

\pagestyle{fancy}
\fancyhf{}
\rhead{Szenzorhalózat}
\chead{Fejlesztői Útmutató}
\lhead{\today}
\cfoot{\thepage}

\title{\textbf{Szenzorhalózat}\\\Large{Fejlesztői Útmutató}}
\author{}
\date{\today}

\begin{document}

\maketitle

\tableofcontents
\newpage

\section{Projekt Áttekintése}

A Szenzorhalózat projekt egy C\# alapú szenzoradatgyűjtési rendszer, amely LiteDB-t használ az adatok persistent tárolásához. A projekt szenzorméréseket kezel és azokat időpecsételt adatbázis-gyűjteményekben tároja.

\section{Projekt Struktúra}

\begin{lstlisting}[language=bash]
Szenzorhalozat/
├── Szenzorhalozat.Core/          # Alaposztályok és üzleti logika
│   ├── Sensors.cs                # Szenzor alapklasszok és implementációk
│   ├── Szenzorhalozat.cs         # Szenzorrendszer vezérlő
│   ├── database.cs               # LiteDB adatbáziskezelés
│   ├── AdatgyujtoAllomas.cs      # Mérési adat gyűjtő
│   └── MeresiAdat.cs             # Mérési adat model
├── Szenzorhalozat.Console/       # Konzol alkalmazás
│   └── Program.cs                # Belépési pont
└── Szenzorhalozat.sln            # Solution fájl
\end{lstlisting}

\section{Főbb Komponensek}

\subsection{1. Sensor -- Absztrakt Szenzor Osztály}

\textbf{Fájl:} \texttt{Szenzorhalozat.Core/Sensors.cs}

\begin{lstlisting}[language=C++]
public abstract class Sensor
{
    [BsonId]
    public int Id { get; set; }
    public string Name { get; set; }
    public string Type { get; protected set; }
    public string Unit { get; protected set; }
    public double CurrentValue { get; protected set; }
    public string Status { get; set; }
    public string CompositeID { get; set; }
    public double[] MinMax = new double[2];
    
    public event System.Action<MeresiAdat>? MeresiAdatKeszult;
    
    public void Meres()           // Mérés trigger
    public void ValueUpd()         // Érték frissítés
    public void UpdateCompositeID() // Composite ID generálás
    protected abstract string StatusUpdate();
}
\end{lstlisting}

\textbf{Kulcsfontosságú Tulajdonságok:}
\begin{itemize}
    \item \texttt{[BsonId]}: LiteDB automatikusan ID-t rendel hozzá
    \item \texttt{MeresiAdatKeszult}: Event, amely mérési adatok rendelkezésre állása esetén aktiválódik
    \item \texttt{MinMax}: Min-max tartomány az értékek generálásához
\end{itemize}

\subsection{2. Szenzor Implementációk}

\textbf{Támogatott Szenzortípusok:}

\begin{table}[h!]
\centering
\begin{tabular}{|l|l|l|l|}
\hline
\textbf{Típus} & \textbf{Osztály} & \textbf{Egység} & \textbf{Tartomány} \\
\hline
Hőmérséklet & \texttt{TemperatureSensor} & °C & 60--120 \\
\hline
Forgás & \texttt{RotationSensor} & RPM & 600--3600 \\
\hline
Vibráció & \texttt{VibrationSensor} & m/s² & 0.5--10 \\
\hline
CO$_2$ & \texttt{CO2Sensor} & PPM & 400--6000 \\
\hline
Nyomás & \texttt{PressureSensor} & bar & 0.5--3 \\
\hline
\end{tabular}
\end{table}

\textbf{Új szenzortípus hozzáadása:}

\begin{lstlisting}[language=C++]
public class CustomSensor : Sensor
{
    public CustomSensor()
    {
        MinMax[0] = 0;      // Min érték
        MinMax[1] = 100;    // Max érték
        
        Name = "Custom Sensor";
        Type = "CUSTOM";
        Unit = "unit";
        CurrentValue = Generate(MinMax[0], MinMax[1]);
        Status = StatusUpdate();
        CompositeID = $"S-{Type.ToUpper()}-{Id:D3}";
    }

    protected override string StatusUpdate()
    {
        if (CurrentValue < 50)
            return "Normal";
        else
            return "Alert";
    }
}
\end{lstlisting}

\subsection{3. Database -- Adatbáziskezelés}

\textbf{Fájl:} \texttt{Szenzorhalozat.Core/database.cs}

\begin{lstlisting}[language=C++]
public class Database : IDisposable
{
    private LiteDatabase db;
    private string currentTableName;
    
    public void AddSensor(Sensor sensor)
    public void GetAllSensors()
    public void AddMeresiAdat(MeresiAdat adat)
    public void GetAllMeresiAdatok()
    public void GetAllTables()
}
\end{lstlisting}

\textbf{Kollekciónevezési konvenció:}
\begin{itemize}
    \item Szenzor adatok: \texttt{T20251210155924} (időpecsétel)
    \item Mérési adatok: \texttt{T20251210155924\_Adatok} (időpecsétel + \texttt{\_Adatok})
\end{itemize}

\subsection{4. Szenzorhalozat -- Rendszer Vezérlő}

\textbf{Fájl:} \texttt{Szenzorhalozat.Core/Szenzorhalozat.cs}

\begin{lstlisting}[language=C++]
public class Szenzorhalozat
{
    public List<Sensor> Szenzorok { get; set; }
    public AdatgyujtoAllomas AdatgyujtoAllomas { get; set; }
    public Database Database { get; set; }
    public event MeresTriggerDelegate? MeresTrigger;
    
    public void SzenzorHozzaadas(Sensor szenzor)
    public void MeresInditas()
}
\end{lstlisting}

\section{Fejlesztési Workflow}

\subsection{1. Projekt Felépítése}

\begin{lstlisting}[language=bash]
cd /home/dev/mnt/szofi/beadando/Szenzorhalozat
dotnet build
\end{lstlisting}

\subsection{2. Függőségek}

\textbf{LiteDB NuGet csomag:}

\begin{lstlisting}[language=XML]
<PackageReference Include="LiteDB" Version="5.0.17" />
\end{lstlisting}

\subsection{3. Hibakeresés}

\begin{lstlisting}[language=bash]
dotnet run --project Szenzorhalozat.Console/Szenzorhalozat.Console.csproj
\end{lstlisting}

\section{Adatáramlás}

\begin{lstlisting}[language=bash]
SzenzorHozzaadas()
    ↓
Database.AddSensor()  [Szenzor tárolás]
    ↓
MeresTrigger event handler regisztrálás
    ↓
MeresInditas() → Meres() hívás [5x]
    ↓
Sensor.MeresiAdatKeszult event
    ↓
AdatgyujtoAllomas.MeresiAdatFogadas()
    ↓
Database.AddMeresiAdat()  [Mérési adatok tárolása]
\end{lstlisting}

\section{Közös Hibák és Megoldások}

\begin{table}[h!]
\centering
\small
\begin{tabular}{|p{3cm}|p{3cm}|p{5cm}|}
\hline
\textbf{Hiba} & \textbf{Ok} & \textbf{Megoldás} \\
\hline
\texttt{'LiteDB' could not be found} & Hiányzó NuGet csomag & \texttt{dotnet restore} \\
\hline
\texttt{Cannot insert duplicate key} & ID ütközés & Statikus ID számlálóval volt probléma \\
\hline
\texttt{MinMax null reference} & Nincs inicializálva & \texttt{MinMax = new double[2]} az ősosztályban \\
\hline
\texttt{CurrentValue = 0} & MaxMin helyett MinMax & Szenzor konstruktorban beállítás \\
\hline
\end{tabular}
\end{table}

\section{Bővítési Lehetőségek}

\begin{enumerate}
    \item \textbf{Adatbázis Migrációk:} Régebbi adatbázis formátumokhoz támogatás
    \item \textbf{Valós Szenzor Integráció:} Fizikai szenzorművek összekötése
    \item \textbf{REST API:} Web interfész a szenzoradatokhoz
    \item \textbf{Grafikonok:} Mérési adatok vizualizációja
    \item \textbf{Riasztások:} Küszöbérték túllépés esetén értesítések
    \item \textbf{Több Felhasználó:} Összes szenzor és futtatás kezelése
\end{enumerate}

\section{Kódolási Irányelvek}

\begin{itemize}
    \item \textbf{Naming:} Magyarul, PascalCase az osztályok és metódusok számára
    \item \textbf{Error Handling:} \texttt{using} utasítás az \texttt{IDisposable} objektumokhoz
    \item \textbf{Events:} Nullable event handlers (\texttt{event Type? EventName})
    \item \textbf{Database:} Mindig \texttt{Dispose()} meghívása vagy \texttt{using} statement
\end{itemize}

\section{Verziózás}

\begin{itemize}
    \item \textbf{Cél Framework:} .NET 9.0
    \item \textbf{C\# verzió:} Latest
    \item \textbf{LiteDB verzió:} 5.0.17
\end{itemize}

\newpage

\begin{center}
\textit{A dokumentáció végéhez értünk.}
\end{center}

\end{document}
