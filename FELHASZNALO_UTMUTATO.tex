\documentclass[12pt,a4paper,magyar]{article}
\usepackage[utf8]{inputenc}
\usepackage[T1]{fontenc}
\usepackage[margin=1in]{geometry}
\usepackage{xcolor}
\usepackage{listings}
\usepackage{hyperref}
\usepackage{fancyhdr}
\usepackage{graphicx}
\usepackage{array}
\usepackage{booktabs}
\usepackage{babel}

\lstset{
    language=bash,
    basicstyle=\ttfamily\small,
    keywordstyle=\color{blue},
    commentstyle=\color{gray},
    stringstyle=\color{red},
    breaklines=true,
    showstringspaces=false,
    tabsize=4,
    frame=single,
    backgroundcolor=\color{gray!10},
    numbers=left,
    numberstyle=\tiny
}

\pagestyle{fancy}
\fancyhf{}
\rhead{Szenzorhalózat}
\chead{Felhasználói Útmutató}
\lhead{\today}
\cfoot{\thepage}

\title{\textbf{Szenzorhalózat}\\\Large{Felhasználói Útmutató}}
\author{}
\date{\today}

\begin{document}

\maketitle

\tableofcontents
\newpage

\section{Bevezetés}

A Szenzorhalózat egy szenzoradatgyűjtési alkalmazás, amely különböző típusú szenzorokat kezel és azok mérési adatait egy adatbázisban tároja. Az alkalmazás automatikusan gyűjti és tároja a mérési adatokat.

\section{Telepítés és Futtatás}

\subsection{Előfeltételek}

\begin{itemize}
    \item \textbf{Operációs rendszer:} Windows, macOS vagy Linux
    \item \textbf{.NET Runtime:} .NET 9.0 vagy újabb
    \item \textbf{Szabad hely:} Legalább 100 MB
\end{itemize}

\subsection{Telepítés Lépései}

\begin{enumerate}
    \item \textbf{Projekt Letöltése}
    \begin{lstlisting}[language=bash]
cd /home/dev/mnt/szofi/beadando/Szenzorhalozat
    \end{lstlisting}

    \item \textbf{Függőségek Telepítése}
    \begin{lstlisting}[language=bash]
dotnet restore
    \end{lstlisting}

    \item \textbf{Fordítás}
    \begin{lstlisting}[language=bash]
dotnet build
    \end{lstlisting}

    \item \textbf{Futtatás}
    \begin{lstlisting}[language=bash]
dotnet run --project Szenzorhalozat.Console/\
  Szenzorhalozat.Console.csproj
    \end{lstlisting}
\end{enumerate}

\subsection{Gyors Indítás}

\begin{lstlisting}[language=bash]
cd /home/dev/mnt/szofi/beadando/Szenzorhalozat
rm -f Meres.db  # Opcionális: előző adatok törlése
dotnet run --project Szenzorhalozat.Console/\
  Szenzorhalozat.Console.csproj
\end{lstlisting}

\section{Használat}

\subsection{A Program Futtatása}

Az alkalmazás indításakor az alábbi lépések történnek:

\begin{enumerate}
    \item \textbf{Szenzor Inicializálása}
    \begin{lstlisting}[language=bash]
Name: Temperature Sensor, Type: TEMP, Value: 116.13 °C, CompID: S-TEMP-001
Name: Temperature Sensor, Type: TEMP, Value: 112.33 °C, CompID: S-TEMP-002
    \end{lstlisting}

    \item \textbf{Mérések Indítása}
    \begin{itemize}
        \item Az alkalmazás 5 alkalommal indít méréseket
        \item Minden mérés között 1 másodperc szünet
    \end{itemize}

    \item \textbf{Mérési Adatok Megjelenítése}
    \begin{lstlisting}[language=bash]
Szenzor ID: 1, Meres ideje: 12/10/2025 15:59:25, Homerseklet: 93.01
Szenzor ID: 2, Meres ideje: 12/10/2025 15:59:25, Homerseklet: 61.23
    \end{lstlisting}

    \item \textbf{Adatbázis Statisztikája}
    \begin{lstlisting}[language=bash]
Adatbázis táblái:
  T20251210155924 - 2 elem
  T20251210155924_Adatok - 10 elem
    \end{lstlisting}
\end{enumerate}

\section{Szenzor Típusok}

Az alkalmazás az alábbi szenzortípusokat támogatja:

\subsection{1. Hőmérséklet Szenzor}

\begin{itemize}
    \item \textbf{Típusazonosító:} \texttt{TEMP}
    \item \textbf{Mértékegység:} °C (Celsius-fok)
    \item \textbf{Mérési Tartomány:} 60 -- 120 °C
    \item \textbf{Állapotok:}
    \begin{itemize}
        \item Alapjárat: $<$ 90 °C
        \item Terhelés: 90 -- 110 °C
        \item Túlmelegedés: $>$ 110 °C
    \end{itemize}
\end{itemize}

\subsection{2. Forgászszenzor}

\begin{itemize}
    \item \textbf{Típusazonosító:} \texttt{ROT}
    \item \textbf{Mértékegység:} RPM (fordulat/perc)
    \item \textbf{Mérési Tartomány:} 600 -- 3600 RPM
    \item \textbf{Állapotok:}
    \begin{itemize}
        \item Alapjárat: $<$ 900 RPM
        \item Terhelés alatt: 900 -- 3000 RPM
        \item Kritikus: $>$ 3000 RPM
    \end{itemize}
\end{itemize}

\subsection{3. Vibráció Szenzor}

\begin{itemize}
    \item \textbf{Típusazonosító:} \texttt{VIB}
    \item \textbf{Mértékegység:} m/s² (méter/másodperc²)
    \item \textbf{Mérési Tartomány:} 0.5 -- 10 m/s²
    \item \textbf{Állapotok:}
    \begin{itemize}
        \item Normál: $<$ 3 m/s²
        \item Magas: 3 -- 6 m/s²
        \item Kritikus: $>$ 6 m/s²
    \end{itemize}
\end{itemize}

\subsection{4. CO\textsubscript{2} Szenzor}

\begin{itemize}
    \item \textbf{Típusazonosító:} \texttt{CO2}
    \item \textbf{Mértékegység:} PPM (részecske millióban)
    \item \textbf{Mérési Tartomány:} 400 -- 6000 PPM
    \item \textbf{Állapotok:}
    \begin{itemize}
        \item Normál: $<$ 5000 PPM
        \item Kritikus: $>$ 5000 PPM
    \end{itemize}
\end{itemize}

\subsection{5. Nyomás Szenzor}

\begin{itemize}
    \item \textbf{Típusazonosító:} \texttt{PRES}
    \item \textbf{Mértékegység:} bar
    \item \textbf{Mérési Tartomány:} 0.5 -- 3 bar
    \item \textbf{Állapotok:}
    \begin{itemize}
        \item Normál: $<$ 1.5 bar
        \item Terhelés: 1.5 -- 3 bar
        \item Kritikus: $>$ 3 bar
    \end{itemize}
\end{itemize}

\section{Adatbázis}

\subsection{Adatfájl}

\begin{itemize}
    \item \textbf{Helye:} \texttt{Meres.db} (a projekt gyökerében)
    \item \textbf{Formátum:} LiteDB (bináris adatbázis)
    \item \textbf{Méret:} A mérési adatok számától függően növekszik
\end{itemize}

\subsection{Adatbázis Szerkezete}

Az alkalmazás két típusú gyűjteményt használ:

\begin{enumerate}
    \item \textbf{Szenzor Gyűjtemény} (pl. \texttt{T20251210155924})
    \begin{itemize}
        \item Szenzor objektumok
        \item Elemek: 2 (ebben a futtatásban)
    \end{itemize}

    \item \textbf{Mérési Adatok Gyűjtemény} (pl. \texttt{T20251210155924\_Adatok})
    \begin{itemize}
        \item Mérési eredmények
        \item Elemek: 10 (5 mérés $\times$ 2 szenzor)
    \end{itemize}
\end{enumerate}

\subsection{Adatok Tartalmazza}

Minden mérési adat az alábbi információkat tartalmazza:

\begin{lstlisting}
Szenzor ID:     1
Meres ideje:    12/10/2025 15:59:25
Homerseklet:    93.00813511586887  (tárolt érték)
\end{lstlisting}

\section{Közös Forgatókönyvek}

\subsection{Forgatókönyv 1: Egy Futtatás Adatainak Megtekintése}

\begin{enumerate}
    \item Futtasd az alkalmazást: \texttt{dotnet run ...}
    \item Az alkalmazás automatikusan megjeleníti:
    \begin{itemize}
        \item Szenzor adatokat
        \item Mérési eredményeket
        \item Adatbázis statisztikáját
    \end{itemize}
\end{enumerate}

\subsection{Forgatókönyv 2: Újabb Futtatás (új adatok)}

\begin{lstlisting}[language=bash]
dotnet run --project Szenzorhalozat.Console/\
  Szenzorhalozat.Console.csproj
\end{lstlisting}

Az új futtatás:
\begin{itemize}
    \item Egy új időpecsétel-alapú gyűjteményt hoz létre
    \item Az előző futtatások adatai megmaradnak az adatbázisban
    \item Összes futtatás adatai elérhetőek
\end{itemize}

\subsection{Forgatókönyv 3: Adatbázis Törlése (tiszta indítás)}

\begin{lstlisting}[language=bash]
rm -f Meres.db
dotnet run --project Szenzorhalozat.Console/\
  Szenzorhalozat.Console.csproj
\end{lstlisting}

\section{Hibakezelés}

\subsection{Hiba: "Cannot run..."}

\textbf{Ok:} A projekt nem fordul le.

\textbf{Megoldás:}
\begin{lstlisting}[language=bash]
dotnet clean
dotnet restore
dotnet build
\end{lstlisting}

\subsection{Hiba: "Meres.db már létezik"}

\textbf{Ok:} Az előző futtatás adatai még az adatbázisban vannak.

\textbf{Megoldás:} Az új adatok automatikusan hozzáadódnak az meglévő adatokhoz. Ez az elvárt viselkedés!

\subsection{Hiba: Nincs kimenete az alkalmazásnak}

\textbf{Ok:} Az alkalmazás háttérben fut vagy nem indul el.

\textbf{Megoldás:}
\begin{lstlisting}[language=bash]
dotnet clean
dotnet build
dotnet run --project Szenzorhalozat.Console/\
  Szenzorhalozat.Console.csproj
\end{lstlisting}

\section{Kimenete Értelmezése}

\subsection{Szenzor Információ}

\begin{lstlisting}
Name: Temperature Sensor, Type: TEMP, Value: 116.13 °C, CompID: S-TEMP-001
\end{lstlisting}

\begin{itemize}
    \item \texttt{Name}: Szenzor neve
    \item \texttt{Type}: Szenzor típusa
    \item \texttt{Value}: Aktuális érték
    \item \texttt{CompID}: Szenzor azonosító (S-[TYP]-[ID])
\end{itemize}

\subsection{Mérési Kimenet}

\begin{lstlisting}
Meres inditasa...
Szenzor ID: 1, Meres ideje: 12/10/2025 15:59:25, Homerseklet: 93.01
Szenzor ID: 2, Meres ideje: 12/10/2025 15:59:25, Homerseklet: 61.23
\end{lstlisting}

\begin{itemize}
    \item \texttt{Meres inditasa...}: Mérési ciklus kezdete
    \item \texttt{Szenzor ID}: A szenzor azonosítója az adatbázisban
    \item \texttt{Meres ideje}: A mérés időpontja (dátum és idő)
    \item \texttt{Homerseklet}: A mért érték (általánosan minden típusú szenzorra)
\end{itemize}

\subsection{Adatbázis Statisztika}

\begin{lstlisting}
Adatbázis táblái:
  T20251210155924 - 2 elem
  T20251210155924_Adatok - 10 elem
\end{lstlisting}

\begin{itemize}
    \item \texttt{T20251210155924}: Az aktuális futtatás szenzor gyűjteménye (2 szenzor)
    \item \texttt{T20251210155924\_Adatok}: Az aktuális futtatás mérési adatok gyűjteménye (10 mérés)
\end{itemize}

\section{Kérdések és Válaszok}

\subsection{K: Mit történik az előző futtatások adataival?}

V: Az adatbázisban maradnak. Új futtatások minden alkalommal egy új gyűjteményt hoznak létre az aktuális időzónának megfelelő névvel.

\subsection{K: Miért vannak 10 mérési adat, ha 5 mérés van?}

V: Mert 2 szenzor van, és mind a kettő mér minden ciklus alatt: 5 mérés $\times$ 2 szenzor = 10 adat.

\subsection{K: Lehet-e összes szenzort egyidőben futtatni?}

V: Igen! A program adott időpontban mind a szenzorokat mér és azok értékeit tároja.

\subsection{K: Miért más érték az első szenzor értéke minden futtatáskor?}

V: Mert az értékek véletlenszerűen generálódnak a megadott tartomány (min-max) között. Ez szimulál szenzor viselkedést.

\section{Verzió Információ}

\begin{itemize}
    \item \textbf{Alkalmazás Verzió:} 1.0
    \item \textbf{.NET Verzió:} 9.0
    \item \textbf{Adatbázis Formátum:} LiteDB 5.0.17
    \item \textbf{Legutóbbi Frissítés:} 2025. december 10.
\end{itemize}

\newpage

\begin{center}
\Large \textbf{Köszönjük, hogy a Szenzorhalózatot használod!} \\[0.5cm]
\small Jó munkát! 🚀
\end{center}

\end{document}
